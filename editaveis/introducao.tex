\part{Aspectos Gerais}
\chapter[Introdução]{Introdução}


\chapter{Escopo}
Para a realização do projeto Prédio inteligente deve-se definir as especificações do limite que contempla o projeto, para isso foi necessário definir os requisitos e analisá-los e aplicar alguns métodos para facilitar a definição do escopo como o dos 5W e 2H representado abaixo:

1-What -O que será feito ?

2-Who- Por quem será feito ?

3-Where- Onde será feito ?

4-When-Quando será feito?

5-Why-Por que será feito?

6-How-Como será feito ?

7-How Much- Quanto custará?

    Dessa forma o escopo foi definido como sendo o Prédio Sustentável inteligente da Faculdade do Gama(FGA) e pode ser melhor observado na EAP visto que a mesma guia a equipe para o término do projeto,não incluindo geração de energia ou algum sistema tecnológico para outra parte do campus exceto o estacionamento parte fundamental do campus que ainda não foi concluída e pode ser resolvida com ideias que possuem nesse projeto , esse escopo foi decidido com base: no  alto custo financeiro para suprir toda demanda energética do campus, futuras alterações de estruturas e dimensões dos prédios já construídos e pelo tempo de entrega do projeto.Por fim,foi realizado um escopo para que futuramente a Universidade de Brasília pudesse aproveitá-lo para a construção do novo prédio trazendo assim conforto e ferramentas necessárias para todos que necessitam do mesmo para absorver conhecimento.(stakeholders).
    Outra subdivisão do nosso projeto, trata da parte de controle de acesso, que é responsável por monitorar todo o acesso de pessoas feito neste nosso prédio tecnológico dentro da FGA. Levando em conta que o nosso trabalho é feito em uma universidade federal, não podemos fechar completamente o acesso à universidade, entretanto nosso objetivo é selecionar quem terá ou não acesso a partes específicas dentro do novo prédio.

Inicialmente a ideia é restringir uma parte do estacionamento só para alunos matriculados na faculdade, e para que seja liberada a entrada, é necessário a liberação automática mediante apresentação de carteirinha, em seguida, o acesso também será restringido nas salas e laboratórios com trancas que serão acopladas nas portas, no caso de laboratórios e salas com equipamentos mais sofisticados, somente será autorizada a entrada com um professor como responsável pelo uso do ambiente, e no caso das salas simples, também haverá necessidade de um responsável, contudo, poderá ser tanto aluno, monitor ou professor. A frequência também será averiguada mediante carteirinha em um aparelho eletrônico portátil que cada professor possuirá. Essa tecnologia e modo de segurança já é utilizado em várias universidades do mundo e especialmente em algumas faculdades particulares em Brasília e por isso é uma ideia a ser implementada dentro da UnB principalmente por questão de segurança.

Na parte estrutural do projeto serão aplicados materiais inteligentes e sustentáveis para amenizar a produção de resíduos e consequentemente o impacto no meio ambiente. Além disso, serão consideradas algumas adaptações sobre o uso das salas para se possa receber o sistema de automação e também quais serão destinadas a laboratórios ou salas de aula, de acordo com as necessidades dos usuários. Por fim, haverá a alteração da posição dos elementos usados em sala para melhorar o impacto que os mesmos têm no aprendizado.

Para a produção energética foi considerada duas formas para suprir a demanda energética da FGA a primeira e principal é a geração de energia por meio de placas fotovoltaicas e a segunda sendo utilizada como reserva será por meio de um gerador movido a biodiesel.

\chapter{Requisitos\label{ch:requisitos}}
\section{Backlog do Produto}
Conforme discutido no capítulo \ref{ch:requisitos}, o \textit{backlog} do produto foi definido com os requisitos agrupados em uma rastreabilidade vertical, seguindo do mais abstrato (ou alto nível) para o mais específico (ou baixo nível).

% \begin{figure}[!h]
%   \centering
%   	\includegraphics[width=0.9\textwidth]{figuras/backlog.eps}
%    \caption{Backlog do Produto\label{fig:backlog}}
% \end{figure}

\chapter{Estudo de Demandas}

\chapter{Estudo de Riscos}
\section{Introdução}
Segundo o \cite{pmbok}  , a Gerência de Riscos de Projeto inclui os processos de planejamento, identificação,
análise, planejamento de respostas e controle de riscos de um projeto. Essa gerência tem como objetivos aumentar a
probabilidade e o impacto dos eventos positivos, e reduzir a probabilidade e o impacto dos eventos negativos no projeto.

\section{Identificação dos Riscos}
Para a identificação dos riscos foi utilizada a técnica SWOT. Nesta técnica são utilizados 4 campos: Força (\textit{Strengths}),
 Oportunidades (\textit{Opportunities}), Fraqueza (\textit{Weaknesses}) e Ameaças (\textit{Threats}). Os campos Força e Oportunidade permitem uma
 visualização amplificada do projeto ao passo que as Fraquezas e Ameaças devem ser devidamente tratadas por meio do
 gerenciamento de riscos para manter o projeto seguindo corretamente. Os campos são definidos como:

\begin{itemize}
  \item \textbf{Força (\textit{Strengths}):} São elementos internos que representam benefícios para o projeto.
  \item \textbf{Oportunidades (\textit{Opportunities}):} São elementos externos que representam benefícios para o projeto.
  \item \textbf{Fraqueza (\textit{Weaknesses}):} São elementos internos que trazem prejuízos para o projeto.
  \item \textbf{Ameaças (\textit{Threats}):} São elementos externos que trazem prejuízos para o projeto.
\end{itemize}

Segue-se a representação gráfica desta técnica aplicada à este projeto.

\pagebreak

\begin{figure}[!h]
 \centering
 \includegraphics[keepaspectratio=true,scale=0.23]{figuras/swot.eps}
 \caption{Técnica SWOT}
\end{figure}

\section{Categorização dos Riscos}
\subsection{Descrição dos Itens da EAR}
\section{Definição de Probabilidade e Impacto dos Riscos}



\chapter{EAP}
A Estrutura Analítica de Projetos (EAP), é uma ferramenta visual que é feita a partir da decomposição das etapas do projeto em ordem cronológica. Ela funciona como um facilitador para a identificação de cada etapa do projeto, facilita os processos de gerenciamento e entregas bem como a estimativa de esforço, custo e duração do mesmo. Além da principal função, a definição do escopo do projeto. A EAP é representada em diagrama, começando do tópico mais geral, em seguida as principais etapas, e por fim as entregas que cada etapa necessita.

Para o projeto do Prédio Inteligente elaborou-se uma EAP para que fosse mais fácil a visualização das etapas que devem ser seguidas, além da definição do escopo do mesmo. Essa ferramenta também serve para que todos da equipe tenham acesso à modo que o projeto será desenvolvido. Sendo assim, as fases principais foram divididas em: Planejamento, Justificativa das Soluções e Viabilidade Econômica. Implicitamente estas fases representam os Pontos de Controle 1, 2 e 3 respectivamente. Consequentemente, foram definidas as entregas que devem ser feitas para cada Ponto de Controle.
 \begin{figure}[!h]
 	\centering
 	\includegraphics[keepaspectratio=true,scale=0.37]{figuras/eap.eps}
 	\caption{EAP do Projeto Prédio Inteligente}
 	\label{fig01}
 \end{figure}

\chapter{Integração entre Áreas}
